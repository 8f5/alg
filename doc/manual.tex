\documentclass{article}

\usepackage{listings}

\lstnewenvironment{shell}{\lstset{%
moredelim=*[is][\itshape]{/@}{@/},
numbers=none,xleftmargin=2em,basicstyle=\ttfamily\small}}{}


\begin{document}
\title{Alg User Manual}
\author{Ale\v{s} Bizjak\\
\texttt{Ales.Bizjak@gmail.com}\\
Faculty of Mathematics and Physics\\
University of Ljubljana
\and
Andrej Bauer\\
\texttt{Andrej.Bauer@andrej.com}\\
Faculty of Mathematics and Physics\\
University of Ljubljana}
\maketitle

\section{Introduction}
\label{sec:introduction}

Alg is a program for enumeration of finite models of algebraic
theories. An algebraic theory is given by a signature (a list of
constants and operations) and axioms expressed in first-order logic.
Alg can do the following:
%
\begin{enumerate}
\item list or count all non-isomorphic models of a given theory,
\item list or count all non-isomorphic indecomposable\footnote{A model
  is indecomposable if it cannot be written as a non-trivial product
  of two smaller models.} models of a given theory.
\end{enumerate}
%
Currently alg has the following limitations:
%
\begin{enumerate}
\item only unary and binary operations are accepted,
\item it is assumed that constants denote pairwise distinct elements.
\end{enumerate}
%
This manual describes how to install and use alg. For a quick start
you need Ocaml 3.12 or higher. Compile alg with
%
\begin{shell}
make
\end{shell}
%
and run
%
\begin{shell}
./alg.native --size 8 theories/unital_commutative_ring.th
\end{shell}
%
For usage information type \texttt{./alg.native -help} and for
examples of theories see the \texttt{theories} subdirectory.

\section{Installation}
\label{sec:installation}

\subsection{How to obtain alg}
\label{sec:how-obtain-alg}

Alg is available at \texttt{http://hg.andrej.com/alg/}. You have three
options:
%
\begin{enumerate}
\item download the ZIP file with source code from
  \texttt{http://hg.andrej.com/alg/archive/tip.zip}
\item clone the repository with the Mercurial revision control system:
%
\begin{shell}
hg clone http://hg.andrej.com/alg/
\end{shell}

\end{enumerate}

%
A precompiled executable may be available for your architecture in the
\texttt{precompiled} subdirectory. Visit



\subsection{Compilation and Installation for Linux and MacOS}
\label{sec:comp-under-linux}

To compile alg you need the Make utility and Ocaml 3.12 or
higher.\footnote{Ocaml is available from \texttt{http://www.ocaml.org/}.}
Just type \texttt{make} at the command line. If all goes well
Ocamlbuild will generate a subdirectory \texttt{_build} and in it the
\texttt{alg.native} executable. It will also create a link to
\texttt{_build/alg.native} from the top directory. To test alg type
%
\begin{shell}
./alg.native --count --size 8 theories/group.th
\end{shell}
%
It should tell you within seconds that there are 5 groups of size 8. 

We provided only a very rudimentary installation procedure for alg.
First edit the \texttt{INSTALL\_DIR} setting in \texttt{Makefile} to
set the directory in which alg should be installed, then run
%
\begin{shell}
sudo make install
\end{shell}
%
This will simply copy \texttt{_build/alg.native} to
\texttt{\$(INSTALL\_DIR)/alg}. You may also wish to stash the
\texttt{theories} subdirectory somewhere for future reference.

\end{document}
