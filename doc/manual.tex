\documentclass{article}

\usepackage{fullpage}
\usepackage{url}
\usepackage{listings}

\lstnewenvironment{shell}{\lstset{%
moredelim=*[is][\itshape]{/@}{@/},
numbers=none,xleftmargin=2em,basicstyle=\ttfamily\small}}{}

\lstnewenvironment{alg}{\lstset{%
moredelim=*[is][\itshape]{/@}{@/},
numbers=none,mathescape=true,
xleftmargin=2em,basicstyle=\ttfamily\small}}{}

\begin{document}
\title{Alg User Manual}
\author{Ale\v{s} Bizjak\thanks{\texttt{Ales.Bizjak0@gmail.com},
  Faculty of Mathematics and Physics, University of Ljubljana}
  \and Andrej Bauer\thanks{\texttt{Andrej.Bauer@andrej.com},
    Faculty of Mathematics and Physics, University of Ljubljana}}

\maketitle

\tableofcontents

\section{Introduction}
\label{sec:introduction}

Alg is a program for enumeration of finite models of algebraic
theories. An algebraic theory is given by a signature (a list of
constants and operations) and axioms expressed in first-order
logic.\footnote{Strictly speaking, the axioms of an algebraic theory
  must be equations, but alg can handle all of first-order logic.}
Examples of algebraic theories include groups, lattices, rings,
fields, and many others. Alg can do the following:
%
\begin{itemize}
\item list or count all non-isomorphic models of a given theory,
\item list or count all non-isomorphic indecomposable\footnote{A model
  is indecomposable if it cannot be written as a non-trivial product
  of two smaller models.} models of a given theory.
\end{itemize}
%
Currently alg has the following limitations:
%
\begin{itemize}
\item only unary and binary operations are accepted,
\item it is assumed that constants denote pairwise distinct elements.
\end{itemize}
%
This manual describes how to install and use alg. For a quick start
you need Ocaml 3.11 or newer and the menhir parser generator. Compile
alg with
%
\begin{shell}
make
\end{shell}
%
and run
%
\begin{shell}
./alg.native --size 8 theories/unital_commutative_ring.th
\end{shell}
%
For usage information type \texttt{./alg.native -help} and for
examples of theories see the \texttt{theories} subdirectory.

Alg is released under the open source simplified BSD License, as
detailed in the next section.

\section{Copyright and License}
\label{sec:copyr-license}

\noindent
Copyright {\copyright} 2010, Ale\v{s} Bizjak and Andrej Bauer

\bigskip

\noindent
All rights reserved.

\bigskip

\noindent
Redistribution and use in source and binary forms, with or without
modification, are permitted provided that the following conditions are
met:
%
\begin{itemize}
\item Redistributions of source code must retain the above copyright
  notice, this list of conditions and the following disclaimer.
\item Redistributions in binary form must reproduce the above
  copyright notice, this list of conditions and the following
  disclaimer in the documentation and/or other materials provided with
  the distribution.
\end{itemize}

This software is provided by the copyright holders and contributors
``as is'' and any express or implied warranties, including, but not
limited to, the implied warranties of merchantability and fitness for
a particular purpose are disclaimed. In no event shall the copyright
holder or contributors be liable for any direct, indirect, incidental,
special, exemplary, or consequential damages (including, but not
limited to, procurement of substitute goods or services; loss of use,
data, or profits; or business interruption) however caused and on any
theory of liability, whether in contract, strict liability, or tort
(including negligence or otherwise) arising in any way out of the use
of this software, even if advised of the possibility of such damage.

\section{Installation}
\label{sec:installation}

\subsection{Downloading alg}
\label{sec:how-obtain-alg}

Alg is available at \url{http://hg.andrej.com/alg/}. You have three
options:
%
\begin{enumerate}
\item download the ZIP file with source code from
  \begin{quote}
    \url{http://hg.andrej.com/alg/archive/tip.zip}
  \end{quote}
\item clone the repository with the Mercurial revision control system:
%
\begin{shell}
hg clone http://hg.andrej.com/alg/
\end{shell}
\item download a precompiled executable for your architecture from
  \begin{quote}
    \url{http://hg.andrej.com/alg/file/tip/precompiled}
  \end{quote}
  %
  if one is available. If you choose this option, make sure that you
  still obtain the ZIP file because the \texttt{theories} subdirectory
  contains a number of useful examples.
\end{enumerate}

\subsection{Installation for Linux and MacOS}
\label{sec:comp-under-linux}

\subsubsection{Prerequisites}

To compile alg you need the Make utility, Ocaml 3.11 or newer, and the
menhir parser generator higher. We will assume you have Make. You can
get Ocaml and menhir in several ways:
%
\begin{enumerate}
\item On Ubuntu, install the packages \texttt{ocaml} and
  \texttt{menhir}:
  %
\begin{shell}
sudo apt-get install ocaml menhir    
\end{shell}
  Similar solutions are available on other Linux distributions.
\item On MacOS the easiest way to install Ocaml and menhir is with
  the macports utility:
\begin{shell}
sudo port install ocaml
sudo port install caml-menhir
\end{shell}
\item If you have GODI installed then you already have Ocaml. Install
  menhir with the \texttt{godi\_console} command, if you do not have it yet.
\item Ocaml is also available from
  %
  \begin{quote}
    \url{http://caml.inria.fr/}
  \end{quote}
  %
  and menhir from
  %
  \begin{quote}
    \url{http://pauillac.inria.fr/~fpottier/menhir/}
  \end{quote}
\end{enumerate}

\subsubsection{Compiling to native code}

To compile alg, type \texttt{make} at the command line. If all goes
well ocamlbuild will generate a subdirectory \texttt{\_build} and in
it the \texttt{alg.native} executable. It will also create a link to
\texttt{\_build/alg.native} from the top directory. To test alg type
%
\begin{shell}
./alg.native --count --size 8 theories/group.th
\end{shell}
%
It should tell you within seconds that there are 5 groups of size 8. 

We provided only a very rudimentary installation procedure for alg.
First edit the \texttt{INSTALL\_DIR} setting in \texttt{Makefile} to
set the directory in which alg should be installed, then run
%
\begin{shell}
sudo make install
\end{shell}
%
This will simply copy \texttt{\_build/alg.native} to
\texttt{\$(INSTALL\_DIR)/alg}. You may also wish to stash the
\texttt{theories} subdirectory somewhere for future reference.

\subsubsection{Compiling to bytecode}

If your version of Ocaml does not compile to native code you can try
compiling to bytecode with
%
\begin{shell}
make byte
\end{shell}
%
This will generate a (significantly slower) \texttt{alg.byte} executable.

\subsubsection{Installation without Make}

If you do not have the Make utility (how can that be?) you can compile
alg directly with ocamlbuild:
%
\begin{shell}
ocamlbuild -use-menhir alg.native
\end{shell}
%
To install alg just copy \texttt{\_build/alg.native} to
\texttt{/usr/local/bin/alg} or some other reasonable place.

\subsection{Installation for Microsoft Windows}
\label{sec:comp-inst-micr}

Sorry, this has not been written yet. But if you have Make and Ocaml
3.11 and menhir, you should be able to just follow the instructions
for Linux.

Note that a Windows precompiled executable may be available at
%
\begin{quote}
  \url{http://hg.andrej.com/alg/tip/precompiled/}
\end{quote}

\section{Input}
\label{sec:input}

An alg input file has extension \texttt{.th} and it describes an
algebraic theory. The syntax vaguely follows the syntax of the Coq
proof assistant. A typical input file might look like this:
%
\begin{alg}
# The axioms of a group.
Theory group.
Constant 1.
Unary inv.
Binary *.
Axiom unit_left: 1 * x = x.
Axiom unit_right: x * 1 = x.
Axiom inverse_left: x * inv(x) = 1.
Axiom inverse_right: inv(x) * x = 1.
Axiom associativity: (x * y) * z = x * (y * z).
\end{alg}
%
There is an optional \texttt{Theory} declaration which names the
theory, then we have declarations of constants, unary and binary
operations, and after that there are the axioms. The precise
syntax rules are as follows.

\subsection{Comments}

Comments are written as in Python, i.e., a comment begins with the
\texttt{\#} symbol and includes everything up to the end of line.

\subsection{General syntactic rules}

An alg input file consists of a sequence of declarations
(\texttt{Theory}, \texttt{Constant}, \texttt{Unary}, \texttt{Binary})
and axioms (\texttt{Axiom}, \texttt{Theorem}). Each declaration and
axiom is terminated with a period.

\subsection{The \texttt{Theory} keyword}

You may give a name to your theory with the declaration
%
\begin{alg}
Theory /@theory_name@/.
\end{alg}
%
\emph{at the beginning of the input file}, possibly preceded by
comments and whitespace. The theory name consists of letters, numbers,
and the underscore. If you do not provide a theory name, alg will
deduce one from the file name.

\subsection{Declaration of operations}

The declarations
%
\begin{alg}
Constant /@$c_1$ $c_2$ $\ldots$ $c_k$@/.
Unary /@$u_1$ $u_2$ $\ldots$ $u_m$@/.
Binary /@$b_1$ $b_2$ $\ldots$ $b_n$@/.
\end{alg}
%
are used to declare constants, unary, and binary operations
respectively. You may declare several constants or operations with a
single declaration, or one at a time. You may mix declarations and
axioms, although it is probably a good idea to declare the constants
and operations first.

A constant may be any string of letters, digits and the underscore
character. In particular, a constant may consist just of digits, for
example \texttt{0} or \texttt{1}.

Unary and binary operations may be strings of letters, digits and the
underscore character. For example, if we declare
%
\begin{alg}
Unary inv.
Binary mult.
\end{alg}
%
then we can write expressions like \texttt{mult(x, inv(y))}. It is
even possible to declare operations whose names are strings of digits,
for example:
%
\begin{alg}
Unary 3 ten.
Binary +.
Axiom: 3(3(x)) + x = ten(x).
\end{alg}
%
Alternatively, we can use \emph{infix} and \emph{prefix} operators.
These follow the Ocaml rules for infix and prefix notation. An
operator is a string of symbols
%
\begin{quote}
  \verb.! $ % & * + - / \ : < = > ? @ \^ | ~.
\end{quote}
% $
where:
%
\begin{itemize}
\item a \emph{prefix operator} is one that starts with \texttt{?},
  \texttt{!} or \texttt{\char126}. It can be used as a unary operation.
\item \emph{infix operators} can be used as binary operations and have
  four levels of precedence, listed from lowest to highest:
  \begin{itemize}
    \item left-associative operators starting with \texttt{|}, \texttt{\&}, \texttt{\$}
    \item right-associative operators starting with \texttt{@} and \texttt{\^}
    \item left-associative operators starting with \texttt{+}, \texttt{-},
      and \texttt{\char92}
    \item left-associative operators starting with \texttt{*}, \texttt{/}, and \texttt{\%} 
    \item right-associative operators starting with \texttt{**}.
  \end{itemize}
  %
  An operator $\circ$ is \emph{left-associative} if $x \circ y \circ
  z$ is understood as $(x \circ y) \circ z$, and
  \emph{right-associative} if $x \circ y \circ z$ is understood as $x
  \circ (y \circ z)$. If you look at the above list again, you will
  notice that operators have the expected precedence and
  associativity. However, if you are unsure about precedence, it is
  best to use a couple of extra parentheses.
\end{itemize}

\subsection{Axioms}

An axiom has the form
%
\begin{alg}
Axiom /@[name]@/: /@<formula>@/.
\end{alg}
%
or
%
\begin{alg}
Theorem /@[name]@/: /@<formula>@/.
\end{alg}
%
There is no difference between an axiom and a theorem as far as alg is
concerned. We use \texttt{Axiom} for the actual axioms and
\texttt{Theorem} for statements that are consequences of axioms and
are worth including in the theory because they make alg run faster,
see Section~\ref{sec:optimization}.

The optional \texttt{\textit{[name]}} is a string of of
letters, digits and the underscore characters. The
\texttt{\textit{<formula>}} is a first-order formula built from the
following logical operations, listed in order of increasing precedence:
%
\begin{center}
  \begin{tabular}{rcl}
    $\forall x\, .\, \phi$ & is written as & \texttt{forall $x$, $\phi$}, \\
    $\exists x\, .\, \phi$ & is written as & \texttt{exists $x$, $\phi$}, \\
    $\phi \Leftrightarrow \psi$ & is written as & \texttt{$\phi$ <-> $\psi$} or \texttt{$\phi$ <=> $\psi$},\\
    $\phi \Rightarrow \psi$ & is written as & \texttt{$\phi$ -> $\psi$} or \texttt{$\phi$ => $\psi$},\\
    $\phi \lor \psi$ & is written as & \texttt{$\phi$ {\char92}/ $\psi$} or \texttt{$\phi$ or $\psi$},\\
    $\phi \land \psi$ & is written as & \texttt{$\phi$ /{\char92} $\psi$} or \texttt{$\phi$ and $\psi$},\\
    $\lnot \phi$ & is written as & \texttt{not $\phi$},\\
    $s = t$ & is written as & \texttt{$s$ = $t$},\\
    $s \neq t$ & is written as & \texttt{$s$ <> $t$} or \texttt{$s$ != $t$},\\
    $\top$ and $\bot$ & are written as & \texttt{True} and \texttt{False}, respectively.
  \end{tabular}
\end{center}
%
An iterated quantification $\forall x_1 \,.\, \forall x_2 \,.\, \cdots
\forall x_n \,.\, \phi$ may be written as
%
\begin{center}
\texttt{forall $x_1$ $x_2$ $\ldots$ $x_n$ , $\phi$.}
\end{center}
%
and similarly for $\exists$.

Axioms may contain free variables. Thus we can write just
%
\begin{alg}
Axiom: x + y = y + x.
\end{alg}
%
instead of
\begin{alg}
Axiom: forall x y, x + y = y + x.
\end{alg}
%

\section{Output}
\label{sec:output-files}

\subsection{Description of the output}
\label{sec:description-output}

The output of alg is meant to be self-explanatory. Nevertheless, here
is what the output consists of:
%
\begin{description}
\item[Title:] the name of the theory.
\item[Theory:] the input file, which can be suppressed with %
  \texttt{--no-source} command-line option.
\item[Models:] a list of the models found. This can be
  suppressed with the \texttt{--count} command-line option. Each model
  has a name \texttt{\textit{theory\_name\_$n$\_$m$}}
  where $n$ is the model size and $m$ is the model sequence number. If
  a model can be decomposed, a decomposition into indecomposable
  factors is given.\footnote{Please note that in general such a
    decomposition is \emph{not} unique.} Tables of all the operations
  are displayed.
\item[Counts:] a table showing how many models of each size were found.
  If more than three sizes were considered, alg also provides a URL to
  query the counts at \texttt{http://oeis.org/}, the On-Line
  Encyclopedia of Integer Sequences.
\end{description}

\subsection{Available output formats}
\label{sec:output-formats}

Alg supports several output formats. The default output format is
plain text and it is sent to the screen. You can choose a different
format with the \texttt{--format} command-line option, and you can
send the output to a file with the \texttt{--output} option. If you
specify an output file but no format, alg guesses the correct format
from the output filename. To see which formats are supported by alg,
type \texttt{alg --help}.

\subsection{Output for futher processing}

Alg can output the models in JSON format, which is suitable for
further processing. Select the JSON format by specifying the
commnad-line option \texttt{--format json} or an output file with the
\texttt{.json} extension.

In Python you can import JSON data from a file
\texttt{mystuff.json} like this:
%
\begin{lstlisting}[language=python,basicstyle=\ttfamily\small]
import json
with open('mystuff.json','r') as f:
    mystuff = json.load(f)
\end{lstlisting}
%
In Mathematica just use the \texttt{Import["\textit{mystuff.json}"]}
command. Most other common programming languages have support for
processing JSON data.


\section{Command-line Options}
\label{sec:command-line-options}

Alg is used as
%
\begin{shell}
alg --size /@<sizes>@/ /@[options]@/ /@<theory.th>@/
\end{shell}
%
where \texttt{\textit{theory.th}} is the input file, and the options are:
%
\begin{description}
\item[\texttt{--size \textit{<sizes>}}]
  A comma-separated list of sizes that alg should consider. You can
  also specify a size interval of the form
  \texttt{\texttt{m}-\textit{n}}. For example, \texttt{1,2,5-8} would
  mean that we consider sizes 1, 2, 5, 6, 7, 8.
\item[\texttt{--count}]
  Do not print out the models, just report the counts.
\item[\texttt{--format \textit{<format>}}]
  Output in the given format. Supported formats are \texttt{text},
  \texttt{html}, and \texttt{latex}.
\item[\texttt{--indecomposable}]
  Output only indecomposable models, i.e., those that are not products
  of smaller models.
\item[\texttt{--no-products}]
  Do not try to generate models as products of smaller models. Use
  this option if you know that a model cannot be a product of smaller
  ones. For example, a field can never be a product of two fields. If
  all of your axioms are equations, then you should \emph{not} use
  this option.
\item[\texttt{--no-source}]
  Do not include the theory in the output.
\item[\texttt{--output \textit{<filename>}}]
  Output to the given file rather than to screen.
\item[\texttt{--help}]
  Print help.
\end{description}

\section{How to use alg efficiently}
\label{sec:optimization}

You should always keep in mind the fact that alg performs a brute
force search with a few optimizations. In the worst case its running
time is doubly exponential in the size of the models because there are
$n^{n^2}$ tables for a binary operation on a set of size $n$.

Alg is optimized for \emph{equational} theories, i.e., those whose
axioms are equations (semigroups, monoids, groups, rings, lattices,
etc., but \emph{not} integral domains and fields). Alg takes advantage
of commutativity, associativity and idempotent laws, and to a smaller
extent of other kinds of equational laws, such as absorption and
distributivity.

Alg checks axioms which are not equations, but does not perform any
optimizations based on them. You should have as few non-equational
axioms as possible. Furthermore, you should always push all the
quantifiers inside. For example, instead of
%
\begin{shell}
Axiom: forall x, exists y, x <> 0 -> x * y = 1.
\end{shell}
%
you should write
%
\begin{shell}
Axiom: forall x, x <> 0 -> (exists y, x * y = 1).
\end{shell}
%
The best kind of axioms are those that allow alg to immediately fill in
a whole column or row. Typically these are axioms about neutral
elements, such as $1 \cdot x = x$ and $0 + x = x$. As a rule of thumb,
every such axiom will increase the maximum manageable size by one.

In general alg performs better if it is given more axioms and
theorems, because each additional statement cuts down the
possibilities. Thus you \emph{should} include theorems which already
follow from other axioms. For example:
%
\begin{itemize}
\item state \emph{both} $1 \cdot x = x$ and $x \cdot 1 = x$, even if
  one of them follows from the other,
\item more generally, state all versions of a symmetric equation, even
  if they all follow from one of them,
\item state laws like $0 \cdot x = 0$ (and also $x \cdot 0 = x$), even
  if they follow from other axioms.
\end{itemize}

You should declare as many constants and as few operations as
possible. A typical example is the theory of lattices. For
\emph{finite} structures the following are equivalent theories:
%
\begin{itemize}
\item a lattice with operations $\land$, $\lor$,
\item a bounded lattice with operations $\land$, $\lor$
  and constants $0$, $1$,
\item a $\lor$-semilattice with operation $\lor$ and constant $0$,
\item a bounded $\lor$-semilattice with operation $\lor$
  and constants $0$, $1$.
\end{itemize}
%
The best choice is the last one because it has just fewest operation and
most constants. Indeed, figuring out that there are 53 lattices of size
7 takes 250 times longer with the theory of a lattice than with the
theory of a bounded $\lor$-semilattice.

Lastly, we should mention that alg generates tables in the order in
which the operations are declared. Sometimes it is much easier to
generate tables for one operation than another, so you should experiment
by switching the order of declarations. For example, the theory of a
ring works much faster if addition $+$ is declared before
multiplication $\times$.

\end{document}
