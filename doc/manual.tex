\documentclass{report}

\usepackage{listings}
\usepackage{minitoc}

\lstnewenvironment{shell}{\lstset{%
moredelim=*[is][\itshape]{/@}{@/},
numbers=none,xleftmargin=2em,basicstyle=\ttfamily\small}}{}


\begin{document}
\title{Alg User Manual}
\author{Ale\v{s} Bizjak\\
\texttt{Ales.Bizjak@gmail.com}\\
Faculty of Mathematics and Physics\\
University of Ljubljana
\and
Andrej Bauer\\
\texttt{Andrej.Bauer@andrej.com}\\
Faculty of Mathematics and Physics\\
University of Ljubljana}
\maketitle

\dominitoc
\tableofcontents

\chapter{Introduction}
\label{sec:introduction}

Alg is a program for enumeration of finite models of algebraic
theories. An algebraic theory is given by a signature (a list of
constants and operations) and axioms expressed in first-order
logic.\footnote{Strictly speaking, the axioms of an algebraic theory
  must be equations, but alg can handle all of first-order logic.}
Examples of algebraic theories include groups, lattices, rings,
fields, and many others. Alg can do the following:
%
\begin{itemize}
\item list or count all non-isomorphic models of a given theory,
\item list or count all non-isomorphic indecomposable\footnote{A model
  is indecomposable if it cannot be written as a non-trivial product
  of two smaller models.} models of a given theory.
\end{itemize}
%
Currently alg has the following limitations:
%
\begin{itemize}
\item only unary and binary operations are accepted,
\item it is assumed that constants denote pairwise distinct elements.
\end{itemize}
%
This manual describes how to install and use alg. For a quick start
you need Ocaml 3.12 or higher. Compile alg with
%
\begin{shell}
make
\end{shell}
%
and run
%
\begin{shell}
./alg.native --size 8 theories/unital_commutative_ring.th
\end{shell}
%
For usage information type \texttt{./alg.native -help} and for
examples of theories see the \texttt{theories} subdirectory.

Alg is released under the open source simplified BSD License, as
detailed in the next section.

\chapter{Copyright and License}
\label{cha:copyr-license}

\noindent
Copyright {\copyright} 2010, Ale\v{s} Bizjak and Andrej Bauer

\bigskip

\noindent
All rights reserved.

\bigskip

\noindent
Redistribution and use in source and binary forms, with or without
modification, are permitted provided that the following conditions are
met:
%
\begin{itemize}
\item Redistributions of source code must retain the above copyright
  notice, this list of conditions and the following disclaimer.
\item Redistributions in binary form must reproduce the above
  copyright notice, this list of conditions and the following
  disclaimer in the documentation and/or other materials provided with
  the distribution.
\end{itemize}

This software is provided by the copyright holders and contributors
``as is'' and any express or implied warranties, including, but not
limited to, the implied warranties of merchantability and fitness for
a particular purpose are disclaimed. In no event shall the copyright
holder or contributors be liable for any direct, indirect, incidental,
special, exemplary, or consequential damages (including, but not
limited to, procurement of substitute goods or services; loss of use,
data, or profits; or business interruption) however caused and on any
theory of liability, whether in contract, strict liability, or tort
(including negligence or otherwise) arising in any way out of the use
of this software, even if advised of the possibility of such damage.

\chapter{Installation}
\label{sec:installation}

\section{Downloading alg}
\label{sec:how-obtain-alg}

Alg is available at \texttt{http://hg.andrej.com/alg/}. You have three
options:
%
\begin{enumerate}
\item download the ZIP file with source code from
  \begin{quote}
    \texttt{http://hg.andrej.com/alg/archive/tip.zip}
  \end{quote}
\item clone the repository with the Mercurial revision control system:
%
\begin{shell}
hg clone http://hg.andrej.com/alg/
\end{shell}
\item download a precompiled executable for your architecture from
  \begin{quote}
    \texttt{http://hg.andrej.com/alg/file/tip/precompiled}
  \end{quote}
  %
  if one is available. If you choose this option, make sure that you
  still obtain the ZIP file because the \texttt{theories} subdirectory
  contains a number of useful examples.
\end{enumerate}

\section{Installation for Linux and MacOS}
\label{sec:comp-under-linux}

To compile alg you need the Make utility and Ocaml 3.12 or
higher.\footnote{Ocaml is available from \texttt{http://www.ocaml.org/}.}
Just type \texttt{make} at the command line. If all goes well
Ocamlbuild will generate a subdirectory \texttt{\_build} and in it the
\texttt{alg.native} executable. It will also create a link to
\texttt{\_build/alg.native} from the top directory. To test alg type
%
\begin{shell}
./alg.native --count --size 8 theories/group.th
\end{shell}
%
It should tell you within seconds that there are 5 groups of size 8. 

We provided only a very rudimentary installation procedure for alg.
First edit the \texttt{INSTALL\_DIR} setting in \texttt{Makefile} to
set the directory in which alg should be installed, then run
%
\begin{shell}
sudo make install
\end{shell}
%
This will simply copy \texttt{\_build/alg.native} to
\texttt{\$(INSTALL\_DIR)/alg}. You may also wish to stash the
\texttt{theories} subdirectory somewhere for future reference.

\section{Installation for Microsoft Windows}
\label{sec:comp-inst-micr}

Sorry, this has not been written yet. But if you have Make and Ocaml
3.12 or alter, you should be able to just follow the instructions for
Linux.

\chapter{The Format of Input Files}
\label{sec:input-files}

\chapter{Command-line Options}
\label{sec:command-line-options}

\chapter{Optimization}
\label{sec:optimization}

\end{document}
